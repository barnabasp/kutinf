
              
\documentclass[a4paper, 12pt]{article}

%% Language and font encodings
\usepackage[magyar]{babel}
\usepackage[utf8x]{inputenc}
\usepackage[T1]{fontenc}
\usepackage{listings}
\usepackage{xcolor}

%% Sets page size and margins
\usepackage[a4paper,top=3cm,bottom=2cm,left=2.5cm,right=2.5cm,marginparwidth=1.75cm]{geometry}
\usepackage{multicol}
\usepackage{parskip}


%% Useful packages
\usepackage{amsmath}
\usepackage{graphicx}
\usepackage[colorinlistoftodos]{todonotes}
\usepackage[colorlinks=true, allcolors=blue]{hyperref}
\usepackage{float}

\begin{document}
\begin{titlepage}   
\begin{center}
\thispagestyle{empty}  

\vspace*{0.7cm}
\rule{\linewidth}{0.5mm} \\[3mm]
\vspace*{0.7cm}

{\LARGE Kutinfo}


\vspace*{0.7cm}
\rule{\linewidth}{0.5mm} \\[3mm]
\rule{\linewidth}{0.5mm} \\[3mm]



{\Large Kő-papír-olló... \\}

\vspace*{0.7cm}
\rule{\linewidth}{0.5mm} \\[3mm]
{\footnotesize Lukovics Márk, Marx Pál Fülöp, Pórfy Barnabás} \\

  \vspace*{2cm}

\begin{figure}[h!]
\begin{center}
\includegraphics[width=0.5\textwidth]{./elte.eps}
\end{center}
\end{figure}
\end{center}
\end{titlepage}

\begin{abstract}
Egyszerű kő-papír-olló játék C++-ban írva, konzolos bevitellel.
\end{abstract}

\section{Bevezetés}

\subsection{Mi ez?}
A kő-papír-olló játékot két ember játszhatja a kezével. 

A játékosok hangosan háromig számolnak vagy a játék nevét, a "kő – papír – olló"-t mondják, minden számolásnál meglendítve a felemelt és ökölbe szorított kezüket. A harmadik számolás után vagy az utána következő ütemben a játékosok kezükkel felveszik a három alakzat egyikét és megmutatják az ellenfelüknek. 


A másik változat a csoportkör, amit 3-6 játékos játszhat, ennek során mindenki megszámolja, hány más játékoson aratott győzelmet az adott körben (mindenki egyszerre mutat). Győzelmeit egy-egy pontként jegyezzük, aki előbb szerez minimum 20 pontot (változó, de ennyi az ajánlott), az a nyertes.

\begin{itemize}
\item kő: a zárt ököl
\item olló: a kinyújtott, szétnyitott mutató és középső ujj
\item papír: a nyitott tenyér
\end{itemize}
\begin{center}
\includegraphics[scale=0.3]{kezjelek}
\end{center}

A játék a legtöbb országban kő, papír, olló-ként népszerű, noha több változat létezik még (pl.: nyúl, puska, répa), és Japánban dzsankennek nevezik, ami tulajdonképpen a kígyó, béka, csiga játék.

Egyes sportokban ezzel a játékkal határozzák meg, melyik csapat kezdi a meccset. Eső esetén néha így döntik el, megtartsák-e a játszmát. Élő szerepjátékokban is használják véletlenek létrehozására, mivel semmiféle felszerelés nem szükséges hozzá. Néha fogadást is kötnek rá, a szerencsejátékokhoz hasonlóan.

\subsection{ Valószínűségek }

A kő-papír ollót néha a pénzfeldobáshoz, a kockadobáshoz vagy a pálcatöréshez hasonlóan egy személy kiválasztásban is használják. Elméletileg mindenkinek ugyanannyi esélye van a győzelemre, de a véletlen választásoktól eltérően azonban itt lehet fejleszteni a játéktechnikát ha több kört játszanak, mert a tapasztalt játékos kiismerheti ellenfele nem-véletlenszerű taktikáját. 

\newpage
\subsection{ Miért éppen ez?}

A választás azért esett erre a játékra, mert ismertünk egy továbbfejlesztett verziót a játékra, amit le akartunk programozni. Ebben a verzióban 15 különböző lehetőség van.

\begin{figure}[H]
\centering
\includegraphics[scale=0.25]{rps15}
\caption{ Kiterjesztett kő-papír-olló}
\label{fig:rps15}
\end{figure}

\section{Tervezet}

\subsection{Hogyan működjön? (Első verzió) }\label{Csom}

Az első ötletünk az volt, hogyha ez a játék kiterjeszthető 15-re, akkor ebből következik, hogy az összes egynél nagyobb pozitív páratlan számra használható. 

Ezért mi egy olyan játékot akartunk megvalósítani, aminek az elején a játékos kiválaszthatja, hány elemű játékkal akar játszani.

\newpage
\subsection{Megvalósítás}
A megvalósításhoz C++ nyelvet használtuk.
\begin{figure}[H]
\centering
\includegraphics[scale=0.5]{owncode}
\caption{ 2n+1-es kő-papír-olló}
\label{fig:rps2n1}
\end{figure}

Amikor itt tartottunk, akkor rájöttünk, hogy ezen az úton elindulva a játék megvalósítása túl sok időt venne igénybe. Emiatt úgy döntöttünk, hogy egy másik irányba terjesztjük ki a játékot. Nevezetesen a többkörös játékra.

\subsection{Hogyan müködjön? (Második verzió) }

A tervünk az volt, hogy az eredeti három lehetőséges kő-papír-olló játékot akárhányszor lehessen játszani egymás után. Ehhez elölről kezdtük a kódot. Ebben a verzióban számolja az eredményeket, viszont itt a lehetőségek kis száma miatt, megtehettük, hogy nevezéktant(rk, pr, sc) használunk. Ami még különbség, hogy itt, szintén a korlátozottság miatt, nem volt szükség arra, hogy a program magának generálja a "kiütési párokat", hanem előre beleírhattuk a kódba.

\section{Kód}
A fent leírt tervezetet az alábbi módon valósítottuk meg:

\begin{lstlisting}[language=C++]
#include <iostream>
#include <stdlib.h>
#include <time.h>
#include <string>

using namespace std;

int rps(int a,int b)
{
    /*
     rock = 0
     paper = 1
     scissors = 2
     */
    if(a==b)
        return 0;
    if((a==1 && b==0) || (a==2 && b==1) || (a==0 && b==2))
        return 1; //user wins
    else
        return 2; //computer wins
}

int main()
{
    int len;
    int user;
    int draw = 0;
    int player = 0;
    int computer = 0;
    string rk = "Rock";
    string pr = "Paper";
    string sc = "Scissors";
    
    string us;
    string co;
    
    cout << "Best out of?" << endl;
    cin >> len;
    if(len < 3)
        len = 3;
    if(len % 2 == 0)
        len = len+1;
    cout << "Best out of: "<< len << endl;
    int len2 = len;
    int comp;
    srand (time(NULL));
    
    
    for(int i = 0; i < len; i++)
    {
        comp  = rand() % 3;
        
        cout << "Rock(0), Paper(1) or Scissors(2)?"<< endl;
        cin >>user;
        while(user < 0 || user > 2)
        {
            cout << "Choose 0, 1 or 2" << endl;
            cin >> user;
        }
        if(comp == 0)
        {
            co = rk;
        }
        if(comp == 1)
        {
            co = pr;
        }
        if(comp == 2)
        {
            co = sc;
        }
        if(user == 0)
        {
            us = rk;
        }
        if(user == 1)
        {
            us = pr;
        }
        if(user == 2)
        {
            us = sc;
        }
        
        cout << "Player: " << us << endl << "Computer: " << co << endl;
        
        if(rps(user,comp) == 0)
        {
            cout << "Draw!" << endl;
            len++;
            draw++;
        }
        if(rps(user,comp) == 1)
        {
            cout << "Player wins round!" << endl;
            player++;
        }
        if(rps(user,comp) == 2)
        {
            cout << "Computer wins round!" << endl;
            computer++;
        }
        cout << " " << endl << "Score: " << endl << "Player: " 
        << player << endl << "Computer: " << computer << endl
        << " " << endl;
        if(computer == len2/2 +1)
        {
            cout << "-----Computer wins!----" << endl;
            cout << " " << endl << "End score: " << endl << "Player: " 
            << player << endl << "Computer: " << computer << endl 
            << " " << endl;
            cout << "----Better luck next time!----";
            return 0;
        }
        if(player == len2/2 +1)
        {
            cout << "----Player wins!----" << endl;
            cout << " " << endl << "End score: ";
        }
    }
    return 0;
}
\end{lstlisting}

\section{Ellenőrzés}

Ez a kód a terveinket megvalósítja. Az viszont egyértelműen nem valósul meg, hogy a gép kitalálja, hogy mit fogunk választani. Tehát ha a gép ellen játszunk, akkor tényleg egyenlő esélyeink van. Ellenben azzal, mint hogyha emberek ellen játszanánk.

\section{Diszkusszió}

Amint azt a bevezetésben írtuk, ezt a játékot több irányba tovább lehet fejleszteni. Az első változat egy érdekesebb megközelítése a játéknak, viszont a megvalósítása nagyobb időigénnyel rendelkezik. A második verzió az egyik egyszerűbb változat, viszont könnyebben megvalósítható.

\section{Felhasznált források}
• https://hu.wikipedia.org/wiki/Kő-pap\%C3\%ADr-olló

• https://tex.stackexchange.com/

• http://www.umop.com/images/rps15.jpg
\end{document}
